% ===Lesson 11: Formatting: fonts and spacing
\documentclass{article}
\usepackage[T1]{fontenc}
\usepackage[margin=1in]{geometry}
\usepackage{verbatim}
\usepackage{xcolor}
\usepackage{soul}
\usepackage[parfill]{parskip}
\usepackage{lipsum}

\begin{document}
\hl{This lesson shows how to change the spacing elements in a document and how to add explicit formatting instructions to the LaTeX source.}
\vspace{0.5cm}

We have already seen that a blank line in your input will generate a new paragraph in LaTeX. This shows up as the paragraph will start with an indent.

\section{Paragraph spacing}

One common style is to have no indents for paragraphs, but instead to have a ‘blank line’ between them. We can achieve that using the \verb|parskip| package.

\begin{verbatim}
\usepackage[parfill]{parskip}
\usepackage{lipsum} % Just for some filler text
\begin{document}
\lipsum
\end{document}    
\end{verbatim}

\lipsum

\section{Forcing a new line}
Most of the time, you should not force a new line in LaTeX: you almost certainly want a new paragraph or to use parskip, as we’ve just seen, to put a ‘blank line’ between paragraphs.

There are a few places where you use \verb|\\| to start a new line without starting a new paragraph:

\begin{itemize}
    \item At the end of table rows
    \item Inside the center environment
    \item In poetry (the verse environment)
\end{itemize}

Almost always, if you are not in one of those special places, you should not use \verb|\\|.

\section{Adding explicit space}
We can insert a thin space (about half the normal thickness) using \verb|\,|. In math mode, there are also other commands: \verb|\.|, \verb|\:| and \verb|\;|, and one for a negative space: \verb|\!|.

Very rarely, for example when creating a title page, you might need to add explicit horizontal or vertical space. We can use \verb|\hspace| and \verb|\vspace| for that.

\section{Explicit text formatting}
We wrote in lesson 3 that most of the time logical structure is preferable. But sometimes you want to make text bold, or italic, or monospaced, etc. There are two types of command for this: ones for short pieces of text, and ones for ‘running’ material.

For short bits of text, we use \verb|\textbf|, \verb|\textit|, \verb|\textrm|, \verb|\textsf|, \verb|\texttt| and \verb|\textsc|.

For running text, we use commands that alter the font setup; the commands here are for example \verb|\bfseries| and \verb|\itshape|. Because these don’t ‘stop’, we need to place them in a group if we want to prevent them from applying to the whole document. LaTeX environments are groups, as are table cells, or we can use \{...\} to make an explicit group.

\newpage
\begin{verbatim}
\begin{document}
Normal text.

{\itshape

This text is italic.

So is this: the effect is not limited to a paragraph.

}
\end{document}
\end{verbatim}
\vspace{0.5cm}
\hrule

Normal text.

{\itshape

This text is italic.

So is this: the effect is not limited to a paragraph.

}
\hrule
\vspace{0.5cm}

We can set font size in a similar way; these commands all work on an ongoing basis. The sizes we set are relative: \verb|\huge|, \verb|\large|, \verb|\normalsize|, \verb|\small| and \verb|\footnotesize| are common. It’s important to finish a paragraph before changing the font size back; see how we add an explicit \verb|\par| (paragraph break) here.

\begin{verbatim}
\begin{center}
{\itshape\large Some text\par}
Normal text
{\bfseries\small Much smaller text\par}
\end{center}
\end{verbatim}

\begin{center}
{\itshape\large Some text\par}
Normal text
{\bfseries\small Much smaller text\par}
\end{center}

\end{document}
% End Lesson 11 (Nov 6 2024)===