% ===Lesson 10: Mathematics
\documentclass{article}
\usepackage[T1]{fontenc}
\usepackage[margin=1in]{geometry}
\usepackage{graphicx}
\usepackage{xcolor}
\usepackage{soul}
\usepackage{verbatim}
\usepackage{indentfirst}
\usepackage{hyperref}
\usepackage{amsmath}

\begin{document}
\hl{This lesson presents LaTeX’s math mode and how you can type inline and display formulas, the extensions provided by the amsmath package, and how to change fonts in math.}
\vspace{0.5cm}

Typesetting complex mathematics is one of the greatest strengths of LaTeX. You can mark up mathematics in a logical way in what is known as ‘math mode’.

\section{Math mode}
In math mode, spaces are ignored and the correct spacing between characters is (almost always) applied. There are two forms of math mode:

\begin{itemize}
    \item inline: using \verb|$<math_notation>$| or \verb|\(<math_notation>\)|
    \item display: using \verb|$$<math_notation>$$| or \verb|\[<math_notation>\]|
\end{itemize}
\vspace{0.5cm}

Here is an inline math: $x + 1 = y$. And here is a display math: $$x + 1 = y$$

You can look up commands for math mode symbols using the \href{https://detexify.kirelabs.org/classify.html}{Detexify} tool.

\section{Display mathematics}
Display math mode is set centered by default and is meant for larger equations that are \textbf{‘part of a paragraph’}. Note that display math environments do not allow a paragraph to end within the mathematics, so you may not have blank lines within the source of the display.

The paragraph should always be started before the display so do not leave a blank line before the display math environment. If you need several lines of mathematics, do not use consecutive display math environments (this produces inconsistent spacing); use one of the multi-line display environments such as \verb|align| from the \verb|amsmath| package described later.

\begin{verbatim}
A paragraph about a larger equation
\[
\int_{-\infty}^{+\infty} e^{-x^2} \, dx
\]
\end{verbatim}

A paragraph about a larger equation
\[
\int_{-\infty}^{+\infty} e^{-x^2} \, dx
\]

You often want a numbered equation, which is created using the equation environment. Let’s try the same example again:

\begin{verbatim}
A paragraph about a larger equation
\begin{equation}
\int_{-\infty}^{+\infty} e^{-x^2} \, dx
\end{equation}
\end{verbatim}

\section{The \texttt{amsmath} package}
Mathematical notation is very rich, and this means that the tools built into the LaTeX kernel can’t cover everything. The \verb|amsmath| package extends the core support to cover a lot more ideas.

\begin{verbatim}
Solve the following recurrence for $ n,k\geq 0 $:
\begin{align*}
  Q_{n,0} &= 1   \quad Q_{0,k} = [k=0];  \\
  Q_{n,k} &= Q_{n-1,k}+Q_{n-1,k-1}+\binom{n}{k}, \quad\text{for $n$, $k>0$.}
\end{align*}
\end{verbatim}

Solve the following recurrence for $ n,k\geq 0 $:
\begin{align*}
  Q_{n,0} &= 1   \quad Q_{0,k} = [k=0];  \\
  Q_{n,k} &= Q_{n-1,k}+Q_{n-1,k-1}+\binom{n}{k}, \quad\text{for $n$, $k>0$.}
\end{align*}

The \verb|align*| environment makes the equations line up on the ampersands, the \& symbols, just like a table. Notice how we’ve used \verb|\quad| to insert a bit of space, and \verb|\text| to put some normal text inside math mode. We’ve also used another math mode command, \verb|\binom|, for a binomial.

Notice that here we used \verb|align*|, and the equation didn’t come out numbered. Most math environments number the equations by default, and the starred variant (with a \verb|*|) disables numbering.
\vspace{0.5cm}

The package also has several other convenient environments, for example for matrices.

AMS matrices.
\[
\begin{matrix}
a & b & c \\
d & e & f
\end{matrix}
\quad
\begin{pmatrix}
a & b & c \\
d & e & f
\end{pmatrix}
\quad
\begin{bmatrix}
a & b & c \\
d & e & f
\end{bmatrix}
\]

\section{Fonts in math mode}
Unlike normal text, font changes in math mode often convey very specific meaning. They are therefore often written explicitly. There are a set of commands you need here:

\begin{itemize}
    \item \verb|\mathrm|: roman (upright)
    \item \verb|\mathit|: italic spaced as ‘text’
    \item \verb|\mathbf|: boldface
    \item \verb|\mathsf|: sans serif
    \item \verb|\mathtt|: monospaced (typewriter)
    \item \verb|\mathbb|: double-struck (blackboard bold) (provided by the amsfonts package)
\end{itemize}

Each of these takes Latin letters as an argument, so for example we might write a matrix as $\mathbf{M}$.
\end{document}
% End Lesson 10 (Nov 5 2024)===